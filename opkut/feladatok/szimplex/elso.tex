\feladatszam Egy üzem kétféle terméket gyárt ($T_1$,$T_2$). A termékek három alkatrész ($A_1$,$A_2$,$A_3$) felhasználásával készülnek. Az első táblázat a termékek szerelési idejét, egységárát és az alkatrészigényüket tartalmazza. Az alkatrészek megmunkálását két gépen végzik ($G_1$,$G_2$). A második táblázat az alkatrészek gépenkénti megmunkálási igényét tartalmazza, és a megmunkálógépek kapacitását. A szerelőüzem kapacitása 220 perc/nap.
\begin{alphanumericlist}
\item Határozza meg a szerelő- és gyártóüzem kapacitását nem meghaladó napi termelést úgy, hogy az árbevétel maximális legyen!
\item Végezzen érzékenységvizsgálatot az 1. termék egységárára illetve a 2. gép kapacitására!
\item Mennyivel kell megváltoztatni a $G_1$ gép kapacitását, hogy az árbevétel 1\%-kal nőjön?
\end{alphanumericlist} 
\begin{align*}
\begin{tabular}{c|ccc|c|c|}
\multicolumn{1}{c}{}&\multicolumn{1}{c}{$A_1$}&\multicolumn{1}{c}{$A_2$}&
\multicolumn{1}{c}{$A_3$}&\multicolumn{1}{c}{Szerelés}&
\multicolumn{1}{c}{Egységár}\\\cline{2-6}
$G_1$&$1$&$0$&$2$&$2$&$27$\\
$G_2$&$0$&$1$&$1$&$1$&$8$\\\cline{2-6}
\end{tabular}&&
\begin{tabular}{c|ccc|c|}
\multicolumn{1}{c}{}&\multicolumn{1}{c}{$A_1$}&\multicolumn{1}{c}{$A_2$}&
\multicolumn{1}{c}{$A_3$}&\multicolumn{1}{c}{Kapacitás}\\\cline{2-5}
$G_1$&$1$&$0$&$1$&$240$\\
$G_2$&$7$&$1$&$1$&$630$\\\cline{2-5}
\end{tabular}
\end{align*}

\begin{megoldas}
Először is a megoldáshoz fel kell írnunk a matematikai modellt, amelyhez ki kell hámoznunk az adatokat a táblázatokból:
\begin{alignat*}{6}
1\cdot 1x_1&+&0\cdot0x_2&+&1\cdot2x_3&+&1\cdot0x_1&+&0\cdot1x_2&+&1\cdot1x_3&\leq240\\
7\cdot1x_1&+&1\cdot0x_2&+&1\cdot2x_3&+&7\cdot0x_1&+&1\cdot1x_2&+&1\cdot1x_3&\leq630\\
&&&&&&&&2x_1&+&1x_2&\leq220\\
&&&\mathrlap{27x_1+8x_2\longrightarrow\max!}
\end{alignat*}

azaz
\begin{alignat*}{3}
3x_1&+&x_2&\leq240\\
9x_1&+&2x_2&\leq630\\
2x_1&+&1x_2&\leq220\\
27x_1&+&8x_2&\longrightarrow\max!\\
x_1&,&x_2&\geq0
\end{alignat*}

A következő lépés az LP feladat sztenderdizálása:
\begin{alignat*}{5}
3x_1&+&x_2&+&u_1&=240\\
9x_1&+&2x_2&+&u_2&=630\\
2x_1&+&1x_2&+&u_3&=220\\
&&\mathclap{-27x_1-8x_2\longrightarrow\min!}\\
&&&&&&\mathllap{x_1,x_2,u_1,u_2,u_3\geq0}
\end{alignat*}

\begin{multicols}{2}
\begin{tabular}{c|cc|c|}
\multicolumn{1}{c}{}&\multicolumn{1}{c}{$x_1$}&
\multicolumn{1}{c}{$x_2$}&\multicolumn{1}{c}{}\\\cline{2-4}
$u_1$&  $2$& \circled{$1$}& $220$\\
$u_2$&  $3$& $1$& $240$\\
$u_3$&  $9$& $2$& $630$\\\cline{2-4}
     & $27$& $8$&   $0$\\\cline{2-4}
\end{tabular}

\begin{tabular}{c|cc|c|}
\multicolumn{1}{c}{}&\multicolumn{1}{c}{$x_1$}&
\multicolumn{1}{c}{$u_1$}&\multicolumn{1}{c}{}\\\cline{2-4}
$x_2$&  $2$&  $1$&   $220$\\
$u_2$&  \circled{$1$}& $-1$&    $20$\\
$u_3$&  $5$& $-2$&   $190$\\\cline{2-4}
     & $11$& $-8$& $-1760$\\\cline{2-4}
\end{tabular}
\end{multicols}\begin{multicols}{2}
\begin{tabular}{c|cc|c|}
\multicolumn{1}{c}{}&\multicolumn{1}{c}{$u_2$}&
\multicolumn{1}{c}{$u_1$}&\multicolumn{1}{c}{}\\\cline{2-4}
$x_2$& $-2$&  $3$&   $180$\\
$x_1$&  $1$& $-1$&    $20$\\
$u_3$& $-5$&  \circled{$3$}&    $90$\\\cline{2-4}
     &$-11$&  $3$& $-1980$\\\cline{2-4}
\end{tabular}

\begin{tabular}{c|cc|c|}
\multicolumn{1}{c}{}&\multicolumn{1}{c}{$u_2$}&
\multicolumn{1}{c}{$u_3$}&\multicolumn{1}{c}{}\\\cline{2-4}
$x_2$&              $3$&           $-1$&    $90$\\
$x_1$&  $-\nicefrac{2}{3}$& $\nicefrac{1}{3}$&    $50$\\
$u_1$&  $-\nicefrac{5}{3}$& $\nicefrac{1}{3}$&    $30$\\\cline{2-4}
     &             $-6$&           $-1$& $-2070$\\\cline{2-4}
\end{tabular}
\end{multicols}

Mivel a $\mathbf{z}-\mathbf{c}$ vektor $\leq0$, ezért megvan az optimális megoldás, amely az $\mathbf{x}=(50,90)$ vektor. A célfüggvény értéke ekkor $-2070$, azonban ez a segédfeladaté! Az eredeti feladatot maximumot keres, így az eredeti célfüggvény értéke $2070$.
\end{megoldas}

\begin{megoldas}
\medskip\hrule\medskip

Az első termék ára $27 \longrightarrow 27+\lambda$. A célfüggvény az alábbi módon változik:
\begin{align*}
(-27-\lambda)x_1-8x_2\longrightarrow\min!&&
(27+\lambda)x_1+8x_2\longrightarrow\max!
\end{align*}

Mivel az $x_1$-hez tartozó paraméter változik, a táblázatban az $x_1$ sorát kell figyelni:
\begin{alignat*}{4}
-6&+&\tfrac{2}{3}\lambda&\leq&0&\qquad\longrightarrow\qquad&\lambda\leq9\\
-1&-&\tfrac{1}{3}\lambda&\leq&0&\qquad\longrightarrow\qquad&\lambda\geq-3\\
\mathclap{z_{\min}=-2070-50\lambda}&&&&&&
\mathclap{z_{\max}=2070+50\lambda}
\end{alignat*}

A második gép kapacitása $630 \longrightarrow 630+\lambda$. Az eredményoszlop eképpen változik:
\begin{alignat*}{5}
90&-&1\lambda&\geq&0&\qquad\longrightarrow\qquad&\lambda&\leq90\\
50&+&\tfrac{1}{3}\lambda&\geq&0&\qquad\longrightarrow\qquad&\lambda&\geq-150\\
30&+&\tfrac{1}{3}\lambda&\geq&0&\qquad\longrightarrow\qquad&\lambda&\geq-90\\
\mathclap{\hskip 0.1cm z_{\min}=-2070-\lambda}&&&&&&&&
\mathclap{\hskip -0.9cm z_{\max}=2070+\lambda}
\end{alignat*}\hrule\medskip

A feladatrész megoldásához először érzékenységvizsgálatot kell végezni. Az első gép kapacitása $240 \longrightarrow 240+\lambda$. Az eredményoszlop az alábbiak szerint alakul:
\begin{alignat*}{5}
90&+&3\lambda&\geq&0&\qquad\longrightarrow\qquad&\lambda&\geq-30\\
50&-&\tfrac{2}{3}\lambda&\geq&0&\qquad\longrightarrow\qquad&\lambda&\leq75\\
30&-&\tfrac{5}{3}\lambda&\geq&0&\qquad\longrightarrow\qquad&\lambda&\leq18\\
\mathclap{\hskip 0.1cm z_{\min}=-2070-6\lambda}&&&&&&&&
\mathclap{\hskip -0.7cm z_{\max}=2070+6\lambda}
\end{alignat*}

Az árbevétel növeléséhez a célfüggvényt értéket kell növelni, tehát:
\begin{align*}
2070+6\lambda&= 2070\cdot1,01\\
\lambda&=3,45
\end{align*}

Ez belefér az érzékenységi intervallumba, a megoldás ekkor $\mathbf{x}=(47,7;100,35)^T$
\end{megoldas}
